% https://github.com/kgeorgiy/olymp.sty
\documentclass[12pt,a4paper,oneside]{article}
\usepackage{cmap}
\usepackage[utf8]{inputenc}
\usepackage[T2A]{fontenc}
\usepackage[english,russian]{babel}
\usepackage[unicode=true, colorlinks=true, linkcolor=blue, urlcolor=blue]{hyperref}
\usepackage[russian]{olymp}
\usepackage{lastpage}
\usepackage{caption}
\usepackage{graphicx}
\usepackage{amsmath}
\usepackage{amssymb}
\usepackage{color}
\usepackage{import}
\usepackage{epigraph}
\usepackage{wrapfig}
\usepackage{verbatim}
\usepackage{listings}
\usepackage{fancyhdr}
\usepackage{datetime} 
\usepackage{tikz}
\usepackage{amsthm,amsmath,amssymb}
\usepackage{mathtools}
\usepackage{tabularx}
\usepackage{cancel}
\usepackage{tcolorbox}
\tcbuselibrary{breakable,external,fitting,hooks,magazine,poster,raster,skins,theorems,vignette,xparse,listings,listingsutf8,minted}

\newcommand{\importproblem}[1]{\import{problems/#1/}{./statement.tex}}
\newcommand{\importtutorial}[1]{\import{problems/#1/}{./tutotial.tex}}

\definecolor{mygray}{rgb}{0.7,0.7,0.7}
\definecolor{ltgray}{rgb}{0.9,0.9,0.9}
\definecolor{fixcolor}{rgb}{0.7,0,0}
\definecolor{red2}{rgb}{0.7,0,0}
\definecolor{dkred}{rgb}{0.4,0,0}
\definecolor{dkblue}{rgb}{0,0,0.6}
\definecolor{dkgreen}{rgb}{0,0.6,0}
\definecolor{brown}{rgb}{0.5,0.5,0}
\newcommand{\green}[1]{{\color{green}{#1}}}
\newcommand{\black}[1]{{\color{black}{#1}}}
\newcommand{\red}[1]{{\color{red}{#1}}}
\newcommand{\dkred}[1]{{\color{dkred}{#1}}}	
\newcommand{\blue}[1]{{\color{blue}{#1}}}
\newcommand{\dkgreen}[1]{{\color{dkgreen}{#1}}}
\newcommand{\yellow}[1]{{\color{yellow}{#1}}}

\newcommand{\LIMZ}{\lim\limits_{x \to 0}}
\newcommand{\LIMNI}{\lim\limits_{n \to \infty}}
\newcommand{\LIMI}{\lim\limits_{x \to \infty}}
\newcommand{\LIM}[1]{\lim\limits_{x \to #1}}
\def\TODO{{\color{red}\bf TODO}}
\def\N{\mathbb{N}}
\def\R{\mathbb{R}}
\def\F2{\mathbb{F}_2}
\def\Z{\mathbb{Z}}
\def\INF{\t{+}\infty}
\def\EPS{\varepsilon}
\def\EMPTY{\varnothing}
\def\PHI{\varphi}
\def\SO{\Rightarrow}
\def\EQ{\Leftrightarrow}
\def\t{\texttt}
\def\c#1{{\rm\sc{#1}}}
\def\O{\mathcal{O}}
\def\NO{\t{\#}}
\def\XOR{\text{ {\raisebox{-2pt}{\ensuremath{\Hat{}}}} }}
\renewcommand{\le}{\leqslant}
\renewcommand{\ge}{\geqslant}
\newcommand{\q}[1]{\langle #1 \rangle}
\newcommand\URL[1]{{\footnotesize{\url{#1}}}}
\newcommand{\sfrac}[2]{{\scriptstyle\frac{#1}{#2}}}
\newcommand{\mfrac}[2]{{\textstyle\frac{#1}{#2}}}

\DeclarePairedDelimiter\ceil{\lceil}{\rceil}
\DeclarePairedDelimiter\floor{\lfloor}{\rfloor}

\binoppenalty=10000
\relpenalty=10000
\exhyphenpenalty=10000

\setcounter{tocdepth}{0}
\setcounter{tocdepth}{1}
\setcounter{tocdepth}{2}
\setcounter{tocdepth}{3}
\setcounter{tocdepth}{4}
\setcounter{tocdepth}{5}

\def\up{\vspace*{-0.5em}}
\def\down{\vspace*{0.5em}}

\renewcommand{\qedsymbol}{$\blacksquare$}
\theoremstyle{definition} % жирный заголовок, плоский текст
\newtheorem{Thm}{\underline{Теорема}}[subsection] % нумерация будет "<номер subsection>.<номер теоремы>"
\newtheorem{Lm}[Thm]{\underline{Lm}} % Нумерация такая же, как и у теорем
\newtheorem{Ex}[Thm]{Упражнение} % Нумерация такая же, как и у теорем
\newtheorem{Code}[Thm]{Код} % Нумерация такая же, как и у теорем
\theoremstyle{plain} % жирный заголовок, курсивный текст
\newtheorem{Def}[Thm]{Def} % Нумерация такая же, как и у теорем
\theoremstyle{remark} % курсивный заголовок, плоский текст
\newtheorem{Cons}[Thm]{Следствие} % Нумерация такая же, как и у теорем
\newtheorem{Conj}[Thm]{Гипотеза} % Нумерация такая же, как и у теорем
\newtheorem{Prop}[Thm]{Утверждение} % Нумерация такая же, как и у теорем
\newtheorem{Rem}[Thm]{Замечание} % Нумерация такая же, как и у теорем
\newtheorem{Remark}[Thm]{Замечание} % Нумерация такая же, как и у теорем
\newtheorem{Algo}[Thm]{Алгоритм} % Нумерация такая же, как и у теорем



% https://tex.stackexchange.com/questions/340708/misplaced-noalign-hline
\def\LINE{\noindent\hrulefill }
% \def\LINE{\vspace*{-1em}\noindent \underline{\hbox to 1\textwidth{{ } \hfil{ } \hfil{ } }}} 


% https://tex.stackexchange.com/questions/8121/how-to-get-new-line-after-subparagraph-title
\usepackage{titlesec}
\titleformat{\subsection}
{\normalfont\normalsize\bfseries}{\thesubsection}{1em}{}
\titlespacing*{\subsection}{\parindent}{3.25ex plus 1ex minus .2ex}{.75ex plus .1ex}

% Indent first paragraph
\usepackage{indentfirst}